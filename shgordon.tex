\documentclass[12pt,a4paper]{report}
 
\usepackage[utf8]{inputenc}
\usepackage{amsmath,amsthm,amssymb}
\usepackage{braket}

 

 
\begin{document}
 

 
\title{Lattice Sinh-Gordon model }
\author{Vincenzo Afferrante} 
 
\maketitle

\chapter{Introduction} 

Our goal is to simulate  a lattice regularized version of sinh-Gordon model.

\chapter{Properties of Sinh-Gordon model in the continuum }

The sinh-Gordon model has many remarkable properties. It is integrable, has a unique vacuum and the simplest symmetry $Z_2$: $\phi -> -\phi$. The action that defines it is \begin{equation}
\mathcal{A} = \int d^2x \left\lbrace\dfrac{1}{2 \pi} \sum_\nu (\partial_\nu \phi )^2 + 2 \mu \cosh(b \phi) \right\rbrace 
\end{equation} Due to his integrability, the S-matrix is factorizable in 2 particles S-matrix, and it is known analitically \begin{equation}
S(\theta) = \dfrac{\sinh(\theta) -i \sin (\pi \beta)}{\sinh(\theta) +i \sin (\pi \beta) }
\end{equation} Here $\theta$ is the rapidity, and it is related to the energy and the momentum by \begin{equation}
E = m \cosh(\theta) \; ; \; P = m \sinh(\theta)
\end{equation} It is introduced the function $\beta(b)$ defined by \begin{equation}
\beta(b) =  \dfrac{b^2}{8\pi} \dfrac{1}{1+ b^2/8 \pi} \,.
\end{equation} This result is obtained by analytic continuation from an analogous calculation in the sine-Gordon model, valid in the "Coleman bound", namely $b < \sqrt{8\pi}$. It is proven that the theory doesn't develop a mass gap over this bound. 

This formula reveals an interesting property, which is not manifest in the Lagrangian, a duality symmetry given by \begin{equation}
b \to 1/b \; ; \; \beta \to 1- \beta \,.
\end{equation} This transformation doesn't change the position of the zeroes in the S-matrix.


\section{Observables}
Calculations of observable quantities have been made for this model.   A notable example is obtained for the quantity $e^{a\phi}$. Namely
\begin{align}
\braket{e^{a \phi}} =  &\left[ \dfrac{m\Gamma(\frac{1}{2+2b^2}) \Gamma(1+ \frac{b^2}{2+2b^2}) }{4 \sqrt{\pi}}  \right]^{-2 a^2} \times  \\ & \exp{\left\lbrace \int_0^\infty \dfrac{dt}{t} \left[ - \dfrac{\sinh^2(2 a b t)}{2 \sinh(b^2t) \sinh(t)\cosh((1+b^2)t) } + 2 a^2 e^{-2t} \right]  \right\rbrace } \nonumber \,,
\end{align} where \begin{equation}
\label{eq:mass_continuum}
m = \dfrac{4 \sqrt{\pi}}{\Gamma(\frac{1}{2+2b^2})\Gamma(1 +\frac{b^2}{2+2b^2}) } \left[ - \dfrac{\mu \pi \Gamma(1+b^2)}{\Gamma(-b^2)} \right]^{\frac{1}{2+2b^2}} \,.
\end{equation} The last equation gives the renormalized particle mass in terms of the bare parameter $\mu$ and the coupling b. It is remarkable that this formula don't satisfy the duality symmetry, and that the mass goes to zero for $b=1$ (which is equivalent to the Coleman bound, after the rescaling of $1/8\pi$ of the kinetic term).

\chapter{Lattice setup}

The lattice action (after rescaling the field as $\phi \to 1/\sqrt{8 \pi} \phi $) is \begin{equation}
\mathcal{A} = a^2 \sum_x \left[ \dfrac{N_d}{8 \pi} \phi(x)^2 + \dfrac{1}{8\pi} \sum_\nu \phi(x) \phi(x+ a \nu) +2  \mu \cosh(b\phi(x)) \right] \,.
\end{equation} We can absorb $a^2$ in the dimensioned parameter $\mu$ defining $\hat \mu = a^2 \mu$. We simulate the model using the grid code, based on a HMC algorithm.  


\chapter{Comparing lattice theory with the continuum }

Simple dimensional analysis show that both the field and the parameter $b$ are dimensionless, while $\mu$ has a dimension 2 in energy. Labelling lattice parameter with an hat, we have $\hat \mu = a^2 \mu$. Naively the continuum is limit is obtained  with $\hat \mu \to 0$. To compare lattice results with the continuum analytic results, we need to find the lines of constant physics.

We can do this using lattice perturbation theory. Calculating the perturbative expansion of the two point function and setting the pole position equal to the mass gives a relation between $ \hat \mu$ and $\hat b$. This can be checked non-perturbatively doing lattice spectroscopy and checking the value of the physical mass obtained.

The theory has an infinite number of interaction terms \begin{align}
\mathcal{A} =  \sum_x &\left[  \phi(x)^2 +  \sum_\nu \phi(x) \phi(x+\nu) +  {\hat \mu b^2}\phi(x)^2 \right. \nonumber \\
  &\left. +   \dfrac{ 2 \hat \mu b^4}{4!} \phi(x)^4 +  \dfrac{2 \hat \mu b^6}{6!} \phi(x)^6 + \dots  \right]
\end{align} which leads to an infinite number of Feynman rules for n-point vertex, with n even.
 
  We can expanding the two point function in perturbation theory over powers of the couple $b^2$. In the continuum both the coupling and the field don't renormalize. There is only a multiplicative renormalization of the mass parameter, given by \eqref{eq:mass_continuum}. Setting the two point function \begin{equation}
  D(p^2) = \dfrac{1}{\hat{p^2}+ m^2} +O(\hat{p^2} +m^2)
\end{equation}  for $p^2 \to -m^2$ one obtains \begin{align}
 \hat m^2=&  \hat \mu b^2 +\dfrac{1}{2}  \hat \mu b^4 I(\hat \mu)+ \dfrac{1}{8} \hat \mu b^6 I(\hat \mu)^2 + O(\hat \mu b^8) \\
 = & \hat m_0 (1 + \frac{1}{2}b^2 I( \hat \mu) + \frac{1}{8} b^4 I(\hat \mu)^2 ) + O(\hat \mu b^8 )
 \end{align} We have the bare mass $\hat m_0^2 =  \hat \mu b^2 $ and the tadpole integral \begin{equation}
 \label{eq:tadpole}
 I(\hat \mu) = \int_{-\pi/a}^{\pi/a} \dfrac{d^2k}{(2 \pi)^2} \dfrac{1}{\hat{k^2} + \hat \mu b^2
  }
 \end{equation} where $\hat{k^2} = \hat{k_x}^2 + \hat{k_y}^2$ and $\hat p = 2 \sin (pa/2)$.
 It immedetialy appears that there is a divergence when one substitutes $l_\mu = a k_\mu$ and takes the limit $a\to 0$. 
 We can substitute $\hat m_0^2$ with $\hat m^2$ in the integrals with an error of order $b^2$. We invert the series \begin{equation}
 \hat m_0^2 = \hat m^2(1 + \frac{1}{2}b^2 I( \hat m) + \frac{1}{8} b^4 I(\hat m)^2 ) + O(\hat m b^6 ) \,.
 \end{equation} This formula gives us, given a physical value of the mass $m$, the bare parameter $m_0$ for a fixed value of the lattice spacing $a$. 
 \section{Evaluation of the tadpole}
 
 We substitute $l_\mu = a k_\mu$ in \eqref{eq:tadpole} obtaining \begin{equation}
 I(\hat m^2) =  \int_{-\pi}^\pi \dfrac{d^2l}{(2 \pi)^2}\dfrac{1}{\hat l^2 + a^2m^2} =  I_1(a^2 m^2)
 \end{equation} The integral has infrared divergencies for $a \to 0$. We can evaluate numerically the value of $I_1$ for $a\to 0$ only dividing the integral in two parts, one within a small radius $\rho$, and the other outside.
 obtaining:
 \begin{align}
 I &= \int_{|l|< \rho} \dfrac{d^2l}{(2 \pi)^2} \dfrac{1}{l^2 + \xi^2} +\int_{|l|> \rho} \dfrac{d^2l}{(2 \pi)^2} \dfrac{1}{\hat l^2 + \xi^2}  \\
 &= \dfrac{1}{4 \pi} \dfrac{\xi^2}{\rho^2} - \dfrac{1}{4 \pi} \ln \left( \dfrac{\rho^2}{\xi^2}\right) + \int_{|l|> \rho} \dfrac{d^2l}{(2 \pi)^2} \dfrac{1}{\hat l^2 } - \xi^2 \int_{|l|> \rho} \dfrac{d^2l}{(2 \pi)^2} \dfrac{1}{(\hat l^2)^2}  +O(\xi^4) \nonumber \\
 &= Z_0 -\xi^2 C_0 -\dfrac{1}{4 \pi} \ln(\xi^2) \nonumber
 \end{align} We defined $\xi^2 = a^2 m^2$ and we introduced the quantities
\begin{align}
Z_0 &= \lim_{\rho \to 0} \int_{|l|> \rho}  \dfrac{d^2l}{(2 \pi)^2} \dfrac{1}{\hat l^2 } + \dfrac{1}{4 \pi} \log(\rho^2)=0.275794 \\
C_0 &= \lim_{\rho \to 0}  \int_{|l|> \rho}  \dfrac{d^2l}{(2 \pi)^2} \dfrac{1}{(\hat l^2)^2 } - \dfrac{1}{4 \pi \rho^2}
\end{align} 
 
 
 
\end{document}