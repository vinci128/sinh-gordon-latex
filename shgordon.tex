\documentclass[12pt,a4paper]{report}
 
\usepackage[utf8]{inputenc}
\usepackage{amsmath,amsthm,amssymb}
\usepackage{braket}

 

 
\begin{document}
 

 
\title{Sinh-Gordon model on lattice}
\author{Vincenzo Afferrante} 
 
\maketitle

\chapter{Introduction} 

Our goal is to simulate  a lattice regularized version of sinh-Gordon model.

\chapter{Properties of Sinh-Gordon model and observables in the continuum }

The sinh-Gordon model has many remarkable properties. It is integrable, has a unique vacuum and the simplest symmetry $Z_2$: $\phi -> -\phi$. The action that defines it is \begin{equation}
\mathcal{A} = \int d^2x \left\lbrace\dfrac{1}{2 \pi} \sum_\nu (\partial_\nu \phi )^2 + 2 \mu \cosh(b \phi) \right\rbrace
\end{equation} Due to his integrability, the S-matrix is factorizable in 2 particles S-matrix, and it is known analitically \begin{equation}
S(\theta) = \dfrac{\sinh(\theta) -i \sin (\pi \beta)}{\sinh(\theta) +i \sin (\pi \beta) }
\end{equation} Here $\theta$ is the rapidity, and it is related to the energy and the momentum by \begin{equation}
E = m \cosh(\theta) \; ; \; P = m \sinh(\theta)
\end{equation} It is introduced the function $\beta(b)$ defined by \begin{equation}
\beta(b) =  \dfrac{b^2}{8\pi} \dfrac{1}{1+ b^2/8 \pi} \,.
\end{equation} This result is obtained by analytic continuation from an analogous calculation in the sine-Gordon model, valid in the "Coleman bound", namely $b < \sqrt{8\pi}$. It is proven that the theory doesn't develop a mass gap over this bound. 



\section{Observables}
Calculations of observable quantities have been made for this model.   A notable example is obtained for the quantity $e^{a\phi}$. Namely
\begin{align}
\braket{e^{a \phi}} =  &\left[ \dfrac{m\Gamma(\frac{1}{2+2b^2}) \Gamma(1+ \frac{b^2}{2+2b^2}) }{4 \sqrt{\pi}}  \right]^{-2 a^2} \times  \\ & \exp{\left\lbrace \int_0^\infty \dfrac{dt}{t} \left[ - \dfrac{\sinh^2(2 a b t)}{2 \sinh(b^2t) \sinh(t)\cosh((1+b^2)t) } + 2 a^2 e^{-2t} \right]  \right\rbrace } \nonumber \,,
\end{align} where \begin{equation}
m = \dfrac{4 \sqrt{\pi}}{\Gamma(\frac{1}{2+2b^2})\Gamma(1 +\frac{b^2}{2+2b^2}) } \left[ - \dfrac{\mu \pi \Gamma(1+b^2)}{\Gamma(-b^2)} \right]^{\frac{1}{2+2b^2}} \,.
\end{equation} The last equation gives the renormalized particle mass in terms of the bare parameter $\mu$ and the coupling b. 

\chapter{Lattice setup}

The lattice action is \begin{equation}
\mathcal{A} = a^2 \sum_x \left[ \dfrac{N_d}{8 \pi} \phi(x)^2 + \dfrac{1}{8\pi} \sum_\nu \phi(x) \phi(x+\nu) +2 \hat \mu \cosh(b\phi(x)) \right] \,.
\end{equation} 


\chapter{Comparing lattice theory with the continuum }

Simple dimensional analysis show that both the field and the parameter $b$ are dimensionless, while $\mu$ has a dimension 2 in energy. Labelling lattice parameter with an hat, we have $\hat \mu = a^2 \mu$. Naively the continuum is limit is obtained  with $\hat \mu \to 0$. To compare lattice results with the continuum analytic results, we need to find the lines of constant physics.

We can do this using lattice perturbation theory. Calculating the perturbative expansion of the two point function and setting the pole position equal to the mass gives a relation between $ \hat \mu$ and $\hat b$. This can be checked non-perturbatively doing lattice spectroscopy and checking the value of the physical mass obtained.
 

 
\end{document}