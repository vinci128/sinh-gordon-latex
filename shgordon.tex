
\documentclass[12pt]{report}
 

\usepackage{amsmath,amsthm,amssymb}
\usepackage{braket}

 

 
\begin{document}
 

 
\title{Sinh-Gordon model on lattice}
\author{Vincenzo Afferrante} 
 
\maketitle

\chapter{Introduction} 

Our goal is to simulate  a lattice regularized version of sinh-Gordon model.

\chapter{Analytical Predictions and observables in the continuum }

Calculations of observable quantities have been made for this model. The action found in literature is \begin{equation}
\mathcal{A} = \int dx^2 \left\lbrace\dfrac{1}{16 \pi} \sum_\nu (\partial_\nu \phi )^2 + 2 \mu \cosh(b \phi) \right\rbrace
\end{equation}  A notable example is obtained for the quantity $e^{a\phi}$. Namely
\begin{align}
\braket{e^{a \phi}} =  &\left[ \dfrac{m\Gamma(\frac{1}{2+2b^2}) \Gamma(1+ \frac{b^2}{2+2b^2}) }{4 \sqrt{\pi}}  \right]^{-2 a^2} \times \nonumber \\ & \exp{\left\lbrace \int_0^\infty \dfrac{dt}{t} \left[ - \dfrac{\sinh^2(2 a b t)}{2 \sinh(b^2t) \sinh(t)\cosh((1+b^2)t) } + 2 a^2 e^{-2t} \right]  \right\rbrace } \,,
\end{align} where \begin{equation}
m = \dfrac{4 \sqrt{\pi}}{\Gamma(\frac{1}{2+2b^2})\Gamma(1 +\frac{b^2}{2+2b^2}) } \left[ - \dfrac{\mu \pi \Gamma(1+b^2)}{\Gamma(-b^2)} \right]^{\frac{1}{2+2b^2}} \,.
\end{equation} The last equation gives the particle mass in the continuum. 

\chapter{Lattice setup}

The lattice action is \begin{equation}
\mathcal{A} = a^2 \sum_x \left[ \dfrac{N_d}{8 \pi} \phi(x)^2 + \dfrac{1}{8\pi} \sum_\nu \phi(x) \phi(x+\nu) +2 \hat \mu \cosh(b\phi(x)) \right] \,.
\end{equation} 


\chapter{Comparing lattice theory with the continuum }

Simple dimensional analysis show that both the field and the parameter $b$ are dimensionless, while $\mu$ has a dimension 2 in energy. Labelling lattice parameter with an hat, we have $\hat \mu = a^2 \mu$. Naively the continuum is limit is obtained  with $\hat \mu \to 0$. To compare lattice results with the continuum analytic results, we need to find the lines of constant physics.

We can do this using lattice perturbation theory. Calculating the perturbative expansion of the two point function and setting the pole position equal to the mass gives a relation between $ \hat \mu$ and $\hat b$. This can be checked non-perturbatively doing lattice spectroscopy and checking the value of the physical mass obtained.
 

 
\end{document}